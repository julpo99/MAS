%! Author = julianramondo
%! Date = 09/11/2023

% Preamble
\documentclass[11pt]{article}
% Packages
\usepackage{amsmath}
\usepackage{multirow}
\usepackage{pgfplots}
\usepackage{float}
\usepackage{caption}
\usepackage[T1]{fontenc}
\usepackage{lmodern}
\usepackage[colorinlistoftodos]{todonotes}
\pgfplotsset{compat=1.12}

\title{Homework 2: Game Theory - Group 42}
\author{Julian Ramondo - 2785746 \and Lukas Unruh - 2772548 \and Mika Rosin - 2817059}
% Document
\begin{document}
    \maketitle

    \setcounter{section}{1}
    \section{Game Theory: Nash equilibrium}

    \subsection{Dining out}
    \subsubsection{Write down the pay-off matrix.}
    \subsubsection{Assuming that they both order simultaneously and without coordinating, what will they order and why?}
    \subsubsection{Alice is quite the romantic type and gets an additional \textit{s} Euros worth of pleasure if they happen to pick the same meal (either both cheap or both expensive). Bob, on the other hand, is a bit of a contrarian and gets an additional amount of pleasure (also equivalent to \textbf{s} Euro) when they happen to favour different meal choices. Assume that $ 0 < s \leq 2 $. How does this change the pay-off matrix and the Nash equilibrium (or equilibria) of this game?}



    \subsection{Hawk versus Dove}

    \subsubsection{Write down the pay-off matrix for this game.}

    \subsubsection{Determine the Nash equilibria for this game and discuss how they change as the cost of aggression (c) increases. Do your results make sense?}




    \subsection{Investment in recycling}

    \subsubsection{Determine each country’s best-response function.}


    \subsubsection{Indicate the pure strategy Nash Equilibrium ($r1^*,r^*$) on the graph;}

    \subsubsection{On your previous figure, show how the equilibrium would change if the intercept of one of the countries’ average benefit functions bi fell from 10 to some smaller number. What would this mean for the recycling efforts of both countries?}


    \subsection{Tragedy of the Commons}

    \subsubsection{Consider the special case where there are only two players (i.e. n = 2). Determine the individual shares x1 and x2 in the Nash equilibrium for this game.}

    \subsubsection{Does this Nash equilibrium optimise social welfare which is the aggregated utility of all players (i.e. u1(x) + u2(x)?}

    \subsubsection{Can you generalise this result to arbitrary n?}

\end{document}

