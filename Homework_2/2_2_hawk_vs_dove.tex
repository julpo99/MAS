\subsection{Hawk versus Dove}

    \subsubsection{Write down the pay-off matrix for this game.}
    Pay-off matrix for animal A, B:
        \begin{table}[h]
            \centering
            \begin{tabular}{llllll}
                &                                                 & \multicolumn{2}{c}{B}                                                                                               & & \\
                &                                                 & \multicolumn{1}{c}{H}                                     & \multicolumn{1}{c}{D}                                   & & \\ \cline{3-4}
                \multirow{2}{*}{A}       & \multicolumn{1}{r|}{H} & \multicolumn{1}{l|}{$0.5*v-c$, $0.5*v-c$} & \multicolumn{1}{l|}{$v$, 0}                             & & \\ \cline{3-4}
                                         & \multicolumn{1}{l|}{D} & \multicolumn{1}{l|}{0, $v$}                               & \multicolumn{1}{l|}{$0.5*v$, $0.5*v$} & & \\ \cline{3-4}
                &                        &                        &                                                           & &
            \end{tabular}
        \end{table}

    \subsubsection{Determine the Nash equilibria for this game and discuss how they change as the cost of aggression ($c$) increases. Do your results make sense?}
        \begin{align*}
            \begin{split}
                p(0.5v-c)+(1-p)v&=0p+(1-p)*0.5v\\
                0.5pv-pc+v-pv &=0.5v-0.5pv\\
                p&=\cfrac{v}{2c}, (\text{symmetric matrix}\Rightarrow p=q)
            \end{split}
        \end{align*}
        This means that the higher the cost of fighting, the less likely the animals will behave like
        hawks. Also, the cost is multiplied by factor two, therefore the probability of Hawk will drop rather
        quickly. However, no matter how big the cost is, with a small probability any animal might still behave like
        a Hawk.

