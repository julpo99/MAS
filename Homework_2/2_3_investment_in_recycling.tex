\subsection{Investment in recycling}

    \subsubsection{Determine each country's best-response function.}
    Since the utilities are defined equally for any country $i$ we can write one best response function for country $i$:
    \begin{equation}
        \begin{split}
            BR_i=(r_j)&=max(b_i(r_i,r_j)r_i-4r_i)\\
            &\Leftrightarrow \cfrac{\partial}{\partial r_i}u_i(r_i,r_j)=0\\
            &\Leftrightarrow \cfrac{\partial}{\partial r_i}((10-r_i+\cfrac{r_j}{2})r_i-4r_i)=0\\
            &\Leftrightarrow \cfrac{\partial}{\partial r_i}10r_i-r^2+\cfrac{r_ir_j}{2}-4r_i=0\\
            &\Leftrightarrow r_i = 3-\cfrac{r_j}{4}
        \end{split}
    \end{equation}

    \subsubsection{Indicate the pure strategy Nash Equilibrium ($r_1^*,r_2^*$) on the graph;}
    \begin{figure}[H]
        \centering
        \begin{tikzpicture}[>=latex]
            \begin{axis}
                [
                axis x line=center,
                axis y line=center,
                width={\linewidth},
                ytick={1,2,...,5},
                xtick={1,2,...,5},
                xlabel={$r_1$},
                ylabel={$r_2$},
                xlabel style={below right},
                ylabel style={above left},
                xmin=0,
                xmax=6,
                ymin=0,
                ymax=6,
                domain=0:6,
                ]
                \addplot+[mark=] {3-(x/4)};
                \addplot+[mark=] {3-(x/4)-(1/2)};
                \addplot+[mark=] {-4*x+12};
                \node[red,scale=6] at (axis cs:12/5,12/5) {.};
                \draw[red, dashed] (axis cs:12/5,12/5) -- (axis cs:12/5,0);
                \node[above left, red] at (axis cs:12/5,0) {$BR_1(r_2)=BR_2(r_1)=\cfrac{12}{5}$};
            \end{axis}
            \node [anchor=north] at (0,0) {$0$};
        \end{tikzpicture}
        \caption{Nash-Equilibrium for country one and two. The Nash-Equilibrium is at $r_i=r_j=\frac{5}{12}$. Additionally, the line where no
        intersection is indicated is the line with different intersect from the next task}
    \end{figure}

    \subsubsection{On your previous figure, show how the equilibrium would change if the intercept of one of the countries' average benefit functions $b_i$ fell from 10 to some smaller number. What would this mean for the recycling efforts of both countries?}
    When calculating the $BR_i$ again, but with $10-c$ instead of 10 in the benefit function, the result is
    $BR_i(r_j)=3-\cfrac{r_j}{4}-\cfrac{c}{2}$. Therefore, the graph would be shifted by half of the applied reduced intercept $c$.
    This would actually decrease the utility of the other country by $\frac{c}{2}$, while the utility at the NE of the county would be
    increased by factor $\cfrac{c}{4}$.

