\subsection{Quantifying the significance of an observed correlation}
Using MC to decide whether the correlation is significant:
\begin{enumerate}
    \item Generate a large number (100000) of samples where the parameters A and S are randomly assigned.
    \item For each random sample, calculate the correlation coefficient between the parameters A and the score S.
    \item Count the number of times the randomly generated correlation coefficients are greater than or equal to the observed correlation coefficient (0.3).
    \item Compute the simulated p-value by computing the proportion of random samples where the correlation coefficient is at least as big as the observed correlation coefficient.
\end{enumerate}

We get a simulated p-value of 0.39302, which generally is generally considered not statistically significant. Therefore, the is correlation is also seen as not statistically significant.