\subsection{Quantifying the significance of an observed correlation}
  To decide whether the correlation is significant; we can use Monte Carlo simulation to compute a simulated p-value
over a large number of random samples.
The steps are as follows:

\begin{enumerate}
    \item We have generated a large number (1000000) of samples where the parameters A and S are randomly assigned.
    \item For each random sample, we calculate the correlation coefficient between the parameters A and the score S
    by using the \texttt{corrcoef} function from the \texttt{numpy} library, which corresponds to the Pearson
    correlation coefficient.
    \item We count the number of times the randomly generated correlation coefficients are greater than or equal to the
    observed correlation coefficient (0.3).
    \item Finally, we compute the simulated p-value by computing the proportion of random samples where the correlation
    coefficient is at least as big as the observed correlation coefficient.
\end{enumerate}

We get a simulated p-value of 0.396591, which is greater than the significance level of 0.05.
Therefore, the correlation is not significant, and we don't know whether we should increase the hyperparameter A or not.