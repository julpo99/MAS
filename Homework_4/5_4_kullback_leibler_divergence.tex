\subsection{Kullback-Leibler divergence}

\subsubsection{3}

Given $f$ and $g$ as normal distributions, the PDFs are the following:
\begin{itemize}
    \item $f(x) = \frac{1}{\sigma \sqrt{2\pi}} e^{-\frac{1}{2}(\frac{x - \mu}{\sigma})^2}$
    \item $g(x) = \frac{1}{\tau \sqrt{2\pi}} e^{-\frac{1}{2}(\frac{x - \nu}{\tau})^2}$
\end{itemize}

We can now substitute these PDFs into the definition of KL divergence and after performing the integration we get: \\
\begin{center}
    $KL(\mu, \sigma, \nu, \tau) = \log(\frac{\tau}{\sigma}) + \frac{\sigma^2 +  (\mu - \nu)^2}{2\tau^2} - \frac{1}{2}$
\end{center}
The formula above is showing that the KL divergence is not symmetric, because the order of the parameters would
change the result. We can therefore conclude that $KL(f \,||\, g) \neq KL(g \,||\, f)$.


This explains why the KL is called a \textbf{divergence} rather than a \textbf{distance}.

\subsubsection{4}
To achieve an accurate result for the MC simulation, we choose a sample size of 1000000.
We then pick the values for
the normal distributions as follows: $\mu=1, \sigma=1, \nu=2, \tau=3$.
Computing the results gives us:
\begin{itemize}
    \item theoretical result: 0.7097233997792209
    \item estimated result: 0.7100029885979247
\end{itemize}

Given the close alignment between the results, it is reasonable to infer that the theoretical result is correct.