\section{MDP 1}
In this task we have two possible actions for each state, which is to go clockwise or couter-clockwise.
We name clockwise as c and couter-clockwise as cc $\Rightarrow A=\{c, cc\}$. Furthermore, the total number of
states is noted as $N$:
\begin{enumerate}
    \item For every node the value function is 10, since there is no discount. Similarly, the q-values are always 10 because there is no discount.
    \item An optimal policy would for example be $\pi(c|s)=1, \pi(cc|s)=0$ for any state $s$ except the absorbing state of course. This is an optimal policy,
    because any state will reach the absorbing state, and will therefore, have a eventual (expected) reward of 10, which is the highest achievable reward. The optimal
    policy is not unique! A second policy, $\pi(c|s)=0, \pi(cc|s)=1$, would also give the same expected rewards. Obvioulsy, there are a lot more possible
    optimal policies. $q*=10$, since we would always reach the absorbing state, without penalty, no matter what next state we chose.
    \item The policy would change, since now the shortest path to the absorbing state has to be taken for the optimal policy. Due to having odd numbers,
    this would also mean that there exists an unique optimal policy, since every state would only have a single shortest path. The optimal policy can be seen in equation
    ~\ref{eqn:policy_c} and~\ref{eqn:policy_cc}:
    \begin{equation}
        \pi(c, s_n)=\left\{
            \begin{array}{ll}
                0\mbox{ if }n<\cfrac{N}{2}\\
                1\mbox{ if }n>\cfrac{N}{2}
            \end{array}
        \right.
        \label{eqn:policy_c}
    \end{equation}
    \begin{equation}
        \pi(cc, s_n)=\left\{
            \begin{array}{ll}
                1\mbox{ if }n<\cfrac{N}{2}\\
                0\mbox{ if }n>\cfrac{N}{2}
            \end{array}
        \right.
        \label{eqn:policy_cc}
    \end{equation}
    Obviously, the value function would also depend on the distance to state 0:
    \begin{align}
        v^*(s_n)=
        \left\{
            \begin{array}{ll}
                10-n &\mbox{ if }n<\cfrac{N}{2}\\
                10-(N-n) &\mbox{ if }n>\cfrac{N}{2}
            \end{array}
        \right.
    \end{align}
    Lastly, the $q*$ value would also depend on the distance to state 0:
    \begin{align}
        q^*(s_n, a)=
        \left\{
            \begin{array}{llll}
                10-n+1 &\mbox{ if }n<\frac{N}{2}\mbox{ and } a=cc\\
                10-n-1 &\mbox{ if }n<\frac{N}{2}\mbox{ and } a=c\\
                10-(N-n)-1 &\mbox{ if }n>\frac{N}{2}\mbox{ and } a=cc\\
                10-(N-n)+1 &\mbox{ if }n>\frac{N}{2}\mbox{ and } a=c
            \end{array}
        \right.
    \end{align}
    \item This would change 2 in a similar way like giving negative rewards, since $\gamma$ would reduce the 
    value of the next state for each state. Instead of reducing the value function of a state s by the distance to state 0 linearly (e.g. by one every step),
    the value function would be reduced by $\gamma^n$ since $\gamma$ would be multiplied with the final reward for
    each step we have to take to the absorbing state.
    \item This would change it similarly to 3, since we have a negative reward for moving again, \textbf{but}, there would not be an unique optimal policy anymore, since there would be at least two strategies (going only left at $s_{N/2}$ or going only right). Actually, the policy
    at state $s_{N/2}$ can be arbitrary, while still being optimal (if it is optimal for the other states).
\end{enumerate}