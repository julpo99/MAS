\section{MDP 2}
\paragraph{1.}
We use the formula given in the lecture $v_\pi(s)=\sum\limits_a\pi(a|s)\sum\limits_{s'}p(s'|s,a)\left[r(s, a, s')+\gamma v_\pi(s')\right]$
to calculate the value function for each state:

\begin{align}
    \begin{split}
        v_\pi(1)&=\cfrac{1}{4}\left[0+\gamma 0\right]+\cfrac{3}{4}\left[-2+\gamma v_\pi(2)\right]\\
                &=-\cfrac{3}{2}+\cfrac{3}{4}v_\pi(2)
    \end{split}
    \label{eqn:v_1}
\end{align}
\begin{align}
    \begin{split}
        v_\pi(2)&=\cfrac{1}{2}\left[-2+\gamma v_\pi(1)\right]+\cfrac{1}{2}\left[-2+v_\pi(3)\right]\\
                &=-1+\cfrac{1}{2}v_\pi(1)-1+\cfrac{1}{2}v_\pi(3)   
    \end{split}
    \label{eqn:v_2}
\end{align}
\begin{align}
    \begin{split}
        v_\pi(3)&=\cfrac{3}{4}\left[-2+v_\pi(2)\right] + \cfrac{1}{4}\left[20+\gamma*0\right]\\
            &=-\cfrac{3}{2}+\cfrac{3}{4}v_\pi(2)+5
    \end{split}
    \label{eqn:v_3}
\end{align}
Given the three equations~\ref{eqn:v_1},~\ref{eqn:v_2}.~\ref{eqn:v_3},
we can solve the equation system and get the following values:
\begin{itemize}
    \item $v_\pi(1)=-4,5$
    \item $v_\pi(2)=-4$
    \item $v_\pi(3)=0.5$
\end{itemize}

\paragraph*{2.}
We use the back-up functions for the q-value:
\begin{itemize}
    \item $q_\pi(2, R)=-2+0.5=-1.5$
    \item $q_\pi(3, L)=-2-4=-6$
\end{itemize}

\paragraph*{3.}
\begin{align}
    \begin{split}
        t(1), v^*&=\max\limits_a\left(
        \begin{bmatrix}
            0 & 0 \\
            0 & -2\\
            -2 & -2\\
            -2 & 20\\
            0 & 0\\
        \end{bmatrix}+
        \begin{bmatrix}
            0 & 0\\
            0 & 0\\
            0 & 0\\
            0 & 0\\
            0 & 0
        \end{bmatrix}
        \right)=
        \begin{bmatrix}
            0\\
            0\\
            -2\\
            20\\
            0
        \end{bmatrix}\\
        t(2), v^*&=\max\limits_a\left(
        \begin{bmatrix}
            0 & 0 \\
            0 & -2\\
            -2 & -2\\
            -2 & 20\\
            0 & 0\\
        \end{bmatrix}+
        \begin{bmatrix}
            0 & 0\\
            0 & -2\\
            0 & 20\\
            -2 & 0\\
            0 & 0
        \end{bmatrix}
        \right)=
        \begin{bmatrix}
            0\\
            0\\
            18\\
            20\\
            0
        \end{bmatrix}\\
    t(3) , v^*&=\max\limits_a\left(
        \begin{bmatrix}
            0 & 0 \\
            0 & -2\\
            -2 & -2\\
            -2 & 20\\
            0 & 0\\
        \end{bmatrix}+
        \begin{bmatrix}
            0 & 0\\
            0 & 18\\
            0 & 20\\
            18 & 0\\
            0 & 0
        \end{bmatrix}
        \right)=
        \begin{bmatrix}
            0\\
            16\\
            18\\
            20\\
            0
        \end{bmatrix}\\
        t(4), v^*&=\max\limits_a\left(
        \begin{bmatrix}
            0 & 0 \\
            0 & -2\\
            -2 & -2\\
            -2 & 20\\
            0 & 0\\
        \end{bmatrix}+
        \begin{bmatrix}
            0 & 0\\
            0 & 18\\
            16 & 20\\
            18 & 0\\
            0 & 0
        \end{bmatrix}
        \right)=
        \begin{bmatrix}
            0\\
            16\\
            18\\
            20\\
            0
        \end{bmatrix}
    \end{split}
\end{align}
\begin{equation}
    \Rightarrow q*=\left(
        \begin{bmatrix}
            0 & 0 \\
            0 & -2\\
            -2 & -2\\
            -2 & 20\\
            0 & 0\\
        \end{bmatrix}+
        \begin{bmatrix}
            0 & 0\\
            0 & 18\\
            16 & 20\\
            18 & 0\\
            0 & 0
        \end{bmatrix}
        \right)=
        \begin{bmatrix}
            0 & 0\\
            0 & 16\\
            14 & 18\\
            16 & 20\\
            0 & 0
        \end{bmatrix}
\end{equation}
From $q^*$ we can see, that there is a unique policy, since there is a unique
maximum value for the successors of each state. The optimal policy is determined by
taking the action leading to the biggest $q*$ value:
\begin{itemize}
    \item $\pi(1|L)=0, \pi(1|R)=1$
    \item $\pi(2|L)=0, \pi(2|R)=1$
    \item $\pi(3|L)=0, \pi(3|R)=1$
\end{itemize}
