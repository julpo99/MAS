%! Author = julianramondo
%! Date = 09/11/2023

% Preamble
\documentclass[11pt]{article}
% Packages
\usepackage{amsmath}
\usepackage{multirow}
\usepackage{pgfplots}
\usepackage{float}
\usepackage{caption}
\usepackage[T1]{fontenc}
\usepackage{lmodern}
\usepackage[colorinlistoftodos]{todonotes}
\pgfplotsset{compat=1.12}

\title{Homework 1: Game Theory - Group 42}
\author{Julian Ramondo - 2785746 \and Lukas Unruh - 2772548 \and Mika Rosin - 2817059}
% Document
\begin{document}
    \maketitle


    \section{Game Theory: Optimality Concepts and Nash Equilibrium}

    \subsection{Odd or even game}

    \subsubsection{Write down the pay-off matrix for this game.}
    \begin{table}[h]
        \centering
        \begin{tabular}{llllll}
            & & \multicolumn{2}{c}{B} & & \\
            &                        & \multicolumn{1}{c}{1}      & 2                          & & \\ \cline{3-4}
            \multirow{2}{*}{A} & \multicolumn{1}{r|}{1} & \multicolumn{1}{l|}{-2, 2} & \multicolumn{1}{l|}{3, -3} &  &  \\ \cline{3-4}
            & \multicolumn{1}{l|}{2} & \multicolumn{1}{l|}{3, -3} & \multicolumn{1}{l|}{-4, 4} & & \\ \cline{3-4}
            &                        &                            &                            & &
        \end{tabular}
    \end{table}

    \subsubsection{What are the regret minimisation strategies in terms of pure strategies?}
    \begin{table}[h]
        \centering
        \begin{tabular}{llllll}
            & & \multicolumn{2}{c}{B} & & \\
            & & \multicolumn{1}{c}{1} & 2 & & \\ \cline{3-4}
            \multirow{2}{*}{A} & \multicolumn{1}{r|}{1} & \multicolumn{1}{l|}{\begin{tabular}[c]{@{}l@{}}
                                                                                  5 0
            \end{tabular}} & \multicolumn{1}{l|}{\begin{tabular}[c]{@{}l@{}}
                                                     0 5
            \end{tabular}} & $\rightarrow$ \textcolor{red}{\fbox{5}}\\ \cline{3-4}
            & \multicolumn{1}{l|}{2} & \multicolumn{1}{l|}{\begin{tabular}[c]{@{}l@{}}
                                                               0 7
            \end{tabular}} & \multicolumn{1}{l|}{\begin{tabular}[c]{@{}l@{}}
                                                     7 0
            \end{tabular}} & $\rightarrow$ \textcolor{red}{7}\\ \cline{3-4}
            & & $\downarrow$ & $\downarrow$ & &\\
            & & \textcolor{red}{7} & \textcolor{red}{\fbox{5}} &
        \end{tabular}
    \end{table}

    Player A should play 1.

    Player B should play 2.

    \subsubsection{What are the regret minimization strategies in terms of mixed strategies?}

    \begin{table}[h]
        \centering
        \begin{tabular}{lllllll}
            & & \multicolumn{2}{c}{B} & & & \\
            & & \multicolumn{1}{c}{\textcolor{blue}{q}}   & \textcolor{blue}{1-q}                      &                      & & \\ \cline{3-5}
            \multirow{2}{*}{A} & \multicolumn{1}{r|}{\textcolor{red}{p}} & \multicolumn{1}{l|}{\begin{tabular}[c]{@{}l@{}}
                                                                                                   5 0
            \end{tabular}} & \multicolumn{1}{l|}{\begin{tabular}[c]{@{}l@{}}
                                                     0 5
            \end{tabular}} & \multicolumn{1}{l|}{\begin{tabular}[c]{@{}l@{}}
                                                     \textcolor{blue}{5-5q}
            \end{tabular}} & & \\ \cline{3-5}
            & \multicolumn{1}{l|}{\textcolor{red}{1-p}} & \multicolumn{1}{l|}{\begin{tabular}[c]{@{}l@{}}
                                                                                  0 7
            \end{tabular}} & \multicolumn{1}{l|}{\begin{tabular}[c]{@{}l@{}}
                                                     7 0
            \end{tabular}} & \multicolumn{1}{l|}{\begin{tabular}[c]{@{}l@{}}
                                                     \textcolor{blue}{7q}
            \end{tabular}} & & \\ \cline{3-5}
            & & \multicolumn{1}{|l|}{\textcolor{red}{5p}} & \multicolumn{1}{l|}{\textcolor{red}{7-7p}} & \multicolumn{1}{l}{} & & \\ \cline{3-4}
            & &                                           &                                            &                      & &
        \end{tabular}
    \end{table}

    \begin{figure}[H]
        \centering
        \begin{tikzpicture}[>=latex]
            \begin{axis}
                [
                axis x line=center,
                axis y line=center,
                width={\linewidth},
                ytick={1,2,...,7},
                xtick={0.1,0.2,...,1},
                xlabel={$p$},
                ylabel={$R_A$},
                xlabel style={below right},
                ylabel style={above left},
                xmin=0,
                xmax=1.1,
                ymin=0,
                ymax=8,
                domain=0:1]
                \addplot+[mark=] {5*x};
                \addplot+[mark=] {7-7*x};
                \node[red,scale=6] at (axis cs:7/12,35/12) {.};
                \draw[red, dashed] (axis cs:7/12,35/12) -- (axis cs:7/12,0);
                \node[above right, red] at (axis cs:7/12,0) {$p=\cfrac{7}{12}$};
            \end{axis}
            \node [anchor=north] at (0,0) {$O$};
        \end{tikzpicture}
        \caption{Player A should play strategy 1 with p = $\cfrac{7}{12}$ and strategy 2 with p = $\cfrac{5}{12}$.}
    \end{figure}

    \begin{figure}[H]
        \begin{tikzpicture}[>=latex][h]
            \centering
            \begin{axis}
                [
                axis x line=center,
                axis y line=center,
                width={\linewidth},
                ytick={1,2,...,5},
                xtick={0.1,0.2,...,1},
                xlabel={$q$},
                ylabel={$R_B$},
                xlabel style={below right},
                ylabel style={above left},
                xmin=0,
                xmax=1.1,
                ymin=0,
                ymax=6,
                domain=0:1]
                \addplot+[mark=] {7*x};
                \addplot+[mark=] {5-5*x)};
                \node[red,scale=6] at (axis cs:5/12,35/12) {.};
                \draw[red, dashed] (axis cs:5/12,35/12) -- (axis cs:5/12,0);
                \node[above right, red] at (axis cs:5/12,0) {$q=\cfrac{5}{12}$};
            \end{axis}
            \node [anchor=north] at (0,0) {$O$};

        \end{tikzpicture}
        \caption{Player B should play strategy 1 with q = $\cfrac{5}{12}$ and strategy 2 with q = $\cfrac{7}{12}$.}
    \end{figure}

    \subsubsection{What are the safety strategies for each player in terms of pure strategies?}

    \begin{table}[h]
        \centering
        \begin{tabular}{llllll}
            & & \multicolumn{2}{c}{B} & & \\
            & & \multicolumn{1}{c}{1} & 2 & & \\ \cline{3-4}
            \multirow{2}{*}{A} & \multicolumn{1}{r|}{1} & \multicolumn{1}{l|}{\begin{tabular}[c]{@{}l@{}}
                                                                                  -2, 2
            \end{tabular}} & \multicolumn{1}{l|}{\begin{tabular}[c]{@{}l@{}}
                                                     3, -3
            \end{tabular}} & $\rightarrow$ \textcolor{red}{\fbox{-2} \hspace{0.2cm}\textbf{MAX}}\\ \cline{3-4}
            & \multicolumn{1}{l|}{2} & \multicolumn{1}{l|}{\begin{tabular}[c]{@{}l@{}}
                                                               3, -3
            \end{tabular}} & \multicolumn{1}{l|}{\begin{tabular}[c]{@{}l@{}}
                                                     -4, 4
            \end{tabular}} &  $\rightarrow$ \textcolor{red}{-4}\\ \cline{3-4}
            & & $\downarrow$ & $\downarrow$ & &\\
            & & \textcolor{red}{\fbox{-3}} & \textcolor{red}{\fbox{-3}} &\\
            & & \textcolor{red}{MAX} & \textcolor{red}{MAX}
        \end{tabular}
    \end{table}


    We can arbitrarily choose between strategy 1 and 2 for player B, since they are both equally good.

    \begin{minipage}{0.1\textwidth}
        \begin{align*}
            v_A^{MAMI} &= -2\\
            v_A^{MAMI} &= 1
        \end{align*}
    \end{minipage}
    \begin{minipage}{0.4\textwidth}
        \begin{align*}
            v_B^{MAMI} &= 1\\
            s_B^{MAMI} &= 1 \textit{ or } 2
        \end{align*}
    \end{minipage}
    \begin{minipage}{0.6\textwidth}
        \begin{align*}
            u_A(1, 1) &= -2 &\textit{   or   } u_A(1, 2) &= 3\\
            u_B(1, 1) &= 2  &\textit{   or   } u_B(1, 2) &= -3
        \end{align*}
    \end{minipage}

    \subsubsection{What are the safety strategies for each player in terms of mixed strategies?}

    \begin{table}[h]
        \centering
        \begin{tabular}{lllllll}
            & & \multicolumn{2}{c}{B} & & & \\
            & & \multicolumn{1}{c}{\textcolor{blue}{q}}      & \textcolor{blue}{1-q}                      &                      & & \\ \cline{3-5}
            \multirow{2}{*}{A} & \multicolumn{1}{r|}{\textcolor{red}{p}} & \multicolumn{1}{l|}{\begin{tabular}[c]{@{}l@{}}
                                                                                                   -2, 2
            \end{tabular}} & \multicolumn{1}{l|}{\begin{tabular}[c]{@{}l@{}}
                                                     3, -3
            \end{tabular}} & \multicolumn{1}{l|}{\begin{tabular}[c]{@{}l@{}}
                                                     \textcolor{blue}{5q-3}
            \end{tabular}} & & \\ \cline{3-5}
            & \multicolumn{1}{l|}{\textcolor{red}{1-p}} & \multicolumn{1}{l|}{\begin{tabular}[c]{@{}l@{}}
                                                                                  3, -3
            \end{tabular}} & \multicolumn{1}{l|}{\begin{tabular}[c]{@{}l@{}}
                                                     -4, 4
            \end{tabular}} & \multicolumn{1}{l|}{\begin{tabular}[c]{@{}l@{}}
                                                     \textcolor{blue}{-7q+4}
            \end{tabular}} & & \\ \cline{3-5}
            & & \multicolumn{1}{|l|}{\textcolor{red}{-5p+3}} & \multicolumn{1}{l|}{\textcolor{red}{7p-4}} & \multicolumn{1}{l}{} & & \\ \cline{3-4}
            & &                                              &                                            &                      & &
        \end{tabular}
    \end{table}

    \begin{figure}[H]
        \begin{tikzpicture}[>=latex]
            \centering
            \begin{axis}
                [
                axis x line=center,
                axis y line=center,
                width={\linewidth},
                ytick={1,2,...,3},
                xtick={0.1,0.2,...,1},
                xlabel={$p$},
                ylabel={$R_A$},
                xlabel style={below right},
                ylabel style={above left},
                xmin=0,
                xmax=1.1,
                ymin=0,
                ymax=4,
                domain=0:1]
                \addplot+[mark=] {-5*x+3};
                \addplot+[mark=] {7*x-4};
                \node[red,scale=6] at (axis cs:7/12,1/12) {.};
                \draw[red, dashed] (axis cs:7/12,1/12) -- (axis cs:7/12,0);
                \node[above right, red] at (axis cs:29/48,0) {$p=\cfrac{7}{12}$};
            \end{axis}
            \node [anchor=north] at (0,0) {$O$};
        \end{tikzpicture}
        \caption{Player A should play strategy 1 with p = $\cfrac{7}{12}$ and strategy 2 with p = $\cfrac{5}{12}$.}
    \end{figure}

    \begin{figure}[H]
        \begin{tikzpicture}[>=latex]
            \centering
            \begin{axis}
                [
                axis x line=center,
                axis y line=center,
                width={\linewidth},
                ytick={-1,0,...,4},
                xtick={0.1,0.2,...,1},
                extra y ticks= {0},
                xlabel={$q$},
                ylabel={$R_B$},
                xlabel style={below right},
                ylabel style={above left},
                xmin=0,
                xmax=1.1,
                ymin=-1,
                ymax=5,
                domain=0:1]
                \addplot+[mark=] {5*x-3};
                \addplot+[mark=] {-7*x+4};
                \node[red,scale=6] at (axis cs:7/12,-1/12) {.};
                \draw[red, dashed] (axis cs:7/12,-1/12) -- (axis cs:7/12,0);
                \node[above, red] at (axis cs:7/12,0.2) {$q=\cfrac{7}{12}$};
            \end{axis}

        \end{tikzpicture}
        \caption{Player B should play strategy 1 with q = $\cfrac{7}{12}$ and strategy 2 with q = $\cfrac{5}{12}$.}
    \end{figure}

    \subsection{Travelers' dilemma: Discrete version}

    \subsubsection{Write down the pay-off matrix for this game.}
    Pay-off matrix for traveler A and B:
    \begin{table}[h]
        \centering
        \begin{tabular}{lllllll}
            & & \multicolumn{3}{c}{B} & & \\
            &                        & \multicolumn{1}{c}{1}         & \multicolumn{1}{c}{2}         & \multicolumn{1}{c}{3}         & & \\ \cline{3-5}
            \multirow{3}{*}{A} & \multicolumn{1}{r|}{1} & \multicolumn{1}{l|}{1, 1}     & \multicolumn{1}{l|}{1+a, 1-a} & \multicolumn{1}{l|}{1+a, 1-a} &  &  \\ \cline{3-5}
            & \multicolumn{1}{l|}{2} & \multicolumn{1}{l|}{1-a, 1+a} & \multicolumn{1}{l|}{2, 2}     & \multicolumn{1}{l|}{2+a, 2-a} & & \\ \cline{3-5}
            & \multicolumn{1}{l|}{3} & \multicolumn{1}{l|}{1-a, 1+a} & \multicolumn{1}{l|}{2-a, 2+a} & \multicolumn{1}{l|}{3, 3} & & \\ \cline{3-5}
        \end{tabular}
    \end{table}

    \subsubsection{Determine the pure Nash equilibriua (PNE, there might be none, one or multiple ones).}
    There are 3 nash equilibria, which occur when the travelers choose the same value. The NEs are marked red in the following pay-off matrix:

    \begin{table}[h]
        \centering
        \begin{tabular}{lllllll}
            & & \multicolumn{3}{c}{B} & & \\
            &                        & \multicolumn{1}{c}{1}                      & \multicolumn{1}{c}{2}                      & \multicolumn{1}{c}{3}                      & & \\ \cline{3-5}
            \multirow{3}{*}{A} & \multicolumn{1}{r|}{1} & \multicolumn{1}{l|}{\textcolor{red}{1, 1}} & \multicolumn{1}{l|}{1+a, 1-a} & \multicolumn{1}{l|}{1+a, 1-a} &  &  \\ \cline{3-5}
            & \multicolumn{1}{l|}{2} & \multicolumn{1}{l|}{1-a, 1+a}              & \multicolumn{1}{l|}{\textcolor{red}{2, 2}} & \multicolumn{1}{l|}{2+a, 2-a} & & \\ \cline{3-5}
            & \multicolumn{1}{l|}{3} & \multicolumn{1}{l|}{1-a, 1+a}              & \multicolumn{1}{l|}{2-a, 2+a} & \multicolumn{1}{l|}{\textcolor{red}{3, 3}} & & \\ \cline{3-5}
        \end{tabular}
    \end{table}

    \subsubsection{Are there mixed Nash equilibria in which all three strategies are mixed?}
    We use the following pay-off matrix with probabilities:

    \begin{table}[h]
        \centering
        \begin{tabular}{lllllll}
            & & \multicolumn{3}{c}{B} & & \\
            &                              & \multicolumn{1}{c}{v}         & \multicolumn{1}{c}{w}         & \multicolumn{1}{c}{1-(v+w)}   & & \\ \cline{3-5}
            \multirow{3}{*}{A} & \multicolumn{1}{r|}{p}       & \multicolumn{1}{l|}{1, 1}     & \multicolumn{1}{l|}{1+a, 1-a} & \multicolumn{1}{l|}{1+a, 1-a} &  &  \\ \cline{3-5}
            & \multicolumn{1}{r|}{q}       & \multicolumn{1}{l|}{1-a, 1+a} & \multicolumn{1}{l|}{2, 2}     & \multicolumn{1}{l|}{2+a, 2-a} & & \\ \cline{3-5}
            & \multicolumn{1}{l|}{1-(p+q)} & \multicolumn{1}{l|}{1-a, 1+a} & \multicolumn{1}{l|}{2-a, 2+a} & \multicolumn{1}{l|}{3, 3} & & \\ \cline{3-5}
        \end{tabular}
    \end{table}

    We then determine $p$ and $q$ using the following equations:
    \begin{equation}
        p+q(1+a)+(1+a)(1-(p+q)) = p(1-a)+2q+(2+a)(1-(p+q))
    \end{equation}
    \begin{equation}
        p+q(1+a)+(1+a)(1-(p+q)) = p(1-a)+q(2-a)+3(1-(p+q))
    \end{equation}

    This results in $p=\frac{1-a+a^2}{a^2+1}$ and $q=\frac{a(1-a)}{a^2+1}$. \\

    To determine whether there are mixed Nash equilibria in which all three strategies are mixed, the values of $p$, $q$ and $1-(p+q)$ have to be not zero. While the actual values of them depend on $a$, we can still verify that it will never be zero:
    \begin{itemize}
        \item $p=0 \implies$ \text{not possible if $a$ is a real number}
        \item $q=0 \implies a=0 \lor a=1$
        \item $1-(p+q)=0 \implies a=0 \lor a=2 $
    \end{itemize}

    Since $0<a<\frac{1}{2}$, neither $p$ $q$ nor $1-(p+q)$ can be zero. Therefore, there are mixed Nash equilibria in which all three strategies are mixed.

    \subsubsection{Does this knowledge about the NE help the travelers in their decision?}
    No, because the value of $a$ is still unknown, and it is therefore not possible to make a statement regarding the NE.

    \subsubsection{Write down the regret matrix. This matrix is similar to the pay-off matrix, but now specifies the regret (rather than the pay-off) for each action profile.}

    \begin{table}[h]
        \centering
        \begin{tabular}{lllllll}
            & & \multicolumn{3}{c}{B} & & \\
            &                        & \multicolumn{1}{c}{1}       & \multicolumn{1}{c}{2}       & \multicolumn{1}{c}{3}       & & \\ \cline{3-5}
            \multirow{3}{*}{A} & \multicolumn{1}{r|}{1} & \multicolumn{1}{l|}{0, 0}   & \multicolumn{1}{l|}{1-a, a} & \multicolumn{1}{l|}{2-a, a} &  &  \\ \cline{3-5}
            & \multicolumn{1}{l|}{2} & \multicolumn{1}{l|}{a, 1-a} & \multicolumn{1}{l|}{0, 0}   & \multicolumn{1}{l|}{1-a, a} & & \\ \cline{3-5}
            & \multicolumn{1}{l|}{3} & \multicolumn{1}{l|}{a, 2-a} & \multicolumn{1}{l|}{a, 1-a} & \multicolumn{1}{l|}{0, 0} & & \\ \cline{3-5}
        \end{tabular}
    \end{table}

    \subsubsection{What are the (pure) regret minimisation strategies?}

    \begin{table}[h]
        \centering
        \begin{tabular}{lllllll}
            & & \multicolumn{3}{c}{B} & & \\
            & & \multicolumn{1}{c}{1} & \multicolumn{1}{c}{2} & \multicolumn{1}{c}{3} & & \\ \cline{3-5}
            \multirow{3}{*}{A} & \multicolumn{1}{r|}{1} & \multicolumn{1}{l|}{0, 0} & \multicolumn{1}{l|}{1-a, a} & \multicolumn{1}{l|}{2-a, a} & $\rightarrow$ \textcolor{red}{2-a}  \\ \cline{3-5}
            & \multicolumn{1}{l|}{2} & \multicolumn{1}{l|}{a, 1-a} & \multicolumn{1}{l|}{0, 0} & \multicolumn{1}{l|}{1-a, a} & $\rightarrow$ \textcolor{red}{1-a} \\ \cline{3-5}
            & \multicolumn{1}{l|}{3} & \multicolumn{1}{l|}{a, 2-a} & \multicolumn{1}{l|}{a, 1-a} & \multicolumn{1}{l|}{0, 0} & $\rightarrow$ \textcolor{red}{\fbox{a}} \\ \cline{3-5}
            & & $\downarrow$ & $\downarrow$ & $\downarrow$ & &\\
            & & \textcolor{red}{2-a} & \textcolor{red}{1-a} & \textcolor{red}{\fbox{a}}
        \end{tabular}
    \end{table}

    Both players should play action 3.

    \subsection{Cournot's Duopoly (continuous version)}

    \subsubsection{What is the best response for each company given the quantity the other company will produce?}
    Utility is given by:
    \begin{equation*}
        \begin{split}
            u_i(q_1,q_2)&=(\alpha -\beta(q_1+q_2) - c_i)*q_i\\
            &=\alpha*q_i -\beta*q_1^2 -\beta*q_2*q_1 - c_i*q_2
        \end{split}
    \end{equation*}
    where either $q_1$ or $q_2$ is fixed and $\alpha, \beta>0$.
    Let's say a company $1$ has to respond to company $2$, then the maximum profit is given by the maximum utility.
    \begin{align*}
        & \cfrac{\partial u_1}{\partial q_1}=\alpha - 2\beta*q_1 - \beta*q_2 - c_1\\
        & \cfrac{\cfrac{\partial u_1}{\partial q_1}}{\partial q_1}=-2\beta
    \end{align*}
    Where we get the maximum by calculating the x-axis intersection of the derivative (while making sure it is a maximum by
    looking at the second derivative)
    \begin{align}
        &\cfrac{\partial u_1}{\partial q_1}=0 \leftrightarrow \alpha- 2\beta*q_1 - \beta*q_2 - c_1=0\\
        & q_1 = \cfrac{-\alpha+\beta*q_2 + c_1}{-2\beta}
        \label{eqn:max_q_1}
    \end{align}
    The value seen in equation~\ref{eqn:max_q_1} is therefore the best response of company one if company two already
    given their quantity. Similarly the best result for company two ($q_2$) can be derived:
    \begin{equation*}
        q_2 = \cfrac{-\alpha+\beta*q_1 + c_1}{-2\beta}
    \end{equation*}

    \subsubsection{Suppose the companies need not decide on their quantity at the same time, but can react to one another (an unlimited number of times). What will be the outcome? (Provide a diagram.)}
    In an infinite game, the two best responses will change for each company until they join in an Equilibrium. The actual values
    will be dependent on the constant parameters $\alpha, \beta, c_1,$ and $c_2$.
    \begin{figure}[h]
        \begin{tikzpicture}[>=latex]
            \centering
            \begin{axis}
                [
                axis x line=center,
                axis y line=center,
                width={\linewidth},
                ytick={},
                xtick={},
                xlabel={$q_2$},
                ylabel={$q_1$},
                xlabel style={below right},
                ylabel style={above left},
                xmin=0,
                xmax=6,
                ymin=0,
                ymax=6,
                domain=0:6]
                \addplot+[mark=] {-2*x+5};
                \addplot+[mark=] {-0.5*x+2.5};
                \node[red,scale=4] at (axis cs:5/3,5/3) {.};
                \node[above right, red] at (axis cs:5/3,5/3) {Nash Equilibrium};
            \end{axis}
        \end{tikzpicture}
        \caption{Illustration of the quantities each company will produce. Notice that the lines are similar with respect to their axes,
            which is only the case if they have the same production cost $c_i$.}
    \end{figure}

    \subsection{Ice cream time!}

    \subsubsection{On a beautiful summer morning Charlize makes her way to the beach and upon arrival finds that her two competitors have already set up their stalls: Alice at location a = 0.1 and Bob at location b = 0.8. Discuss what Charlize’s best response is: i.e., what location should she choose, given a =0.1 and b =0.8?}
    We can give the utility as the area that Charlize will serve (equation~\ref{eqn:utility_c})
    \begin{align}
        \begin{split}
            u_c(a, b)&=(c+((b-c)/2)) - (c+(a-c)/2)\\
            & = 0.5b - 0.5c - 0.5a + 0.5c\\
            & = (b-a)/2
        \end{split}
        \label{eqn:utility_c}
    \end{align}
    Therefore, the area Charlize will cover is maximized as long as she choses her position between Alice and Bob.

    \subsubsection{Same question as above, but now assume that all we know is that $\mathbf{a = 0.1}$ and $\mathbf{a<b\leq1}$.}
    We can make a case distinction. The area we get between Alice and Bob will be exactly half of the area between them.
    Once the area that is available to us if we go in-between Alice and Bob becomes smaller than the area to the right of
    Bob, we just go to the right of Bob.
    Otherwise, we stay in the middle.
    \begin{equation*}
        BR_c(a,b)=\left\{
        \begin{array}{lll}
            & \text{directly right of b } & \mbox{if } \frac{(b-a)}{2}\leq(1-b) \\
            & \mbox{between a and b}      & \text{otherwise}
        \end{array}
        \right.
    \end{equation*}

    \subsubsection{Earlier that morning, Bob arrived and discovered that Alice had already set up her stall at a = 0.1, while Charlize hadn’t shown up yet. But Bob knows that Charlize will arrive before too long, and that she will try to position herself in such a way as to maximize her revenue. What location should Bob pick in order to maximize his own expected revenue?}
    We want to maximize our expected utility (served area) considering that one more seller (Charlize) will also come and maximize
    her expected utility. For this, we need a position in which Charlize will move between us an Alice and the area between us and
    Alice is minimized.
    This is exactly the case (see above) when the following condition is true:
    \begin{equation*}
        1-b = (b-a)/2
    \end{equation*}
    Therefore, Bob will position himself exactly at $0,7$ to maximize his utility while considering that Charlize will still show up.

    \subsubsection{At sunrise that morning, Alice arrived before both Bob and Charlize, and set up her stall at location a =0.1. However, she knows for sure that the other two vendors will show up soon. Where should she have set up her stall in order to maximize her expected revenue?}
    She should set up her place exactly at $\frac{1}{4}$ or $\frac{3}{4}$ (because of symmetry).
    This can be shown by checking what happens
    if Alice does not choose this strategy:
    \paragraph*{Case 1}: If Alice choses a higher Point than $\frac{1}{4}$, then Bob can simply mirror her chosing (as seen from the other side), but moving a tiny bit towards the edge.
    Now, Charlize has three options, move left of Alice, right of Bob or between them.\\
    \begin{itemize}
        \item Between: Charlize will receive $\frac{1}{2}$ of the area between Alice and Charlize. This is less than $\frac{1}{4}$ since Alice
        moved towards the right. Let $\epsilon$ be the value Alice moved right and $\gamma<\epsilon$ be the shift of Bob to the right. If Bob mirrors Alice as described,
        the area between them is: 
        \begin{equation*}
            \begin{split}
            \text{area}&=\cfrac{(b+\gamma)-(a+\epsilon)}{2}=\cfrac{(1-(a+\epsilon)+\gamma)-(a+\epsilon)}{2}\\
            &=\cfrac{1-a-\epsilon+\gamma-a-\epsilon}{2}=\cfrac{1-2*a-2\epsilon+\gamma}{2}\\
            &=\cfrac{1}{2}-a-\cfrac{2\epsilon+\gamma}{2}=\cfrac{1}{4}-\cfrac{2\epsilon-\gamma}{2}
            \end{split}
        \end{equation*}
        Since $a=\frac{1}{4}$ and $\gamma<\epsilon$, the area between Alice and Bob will be less than $\frac{1}{4}$
        \item Right of Bob: The area is $\frac{1}{4}+\epsilon-\gamma$ 
        \item Left of Alice: The area is $\frac{1}{4}+\epsilon$
    \end{itemize}
    $\Rightarrow$The area left of Alice will be the largest, i.e.\ Charlize moves there. Since Charlize moves as close to
    Alice as possible, the remaining area for Alice will be half of the area between her and Bob, which is $<\frac{1}{4}$
    
    \paragraph*{Case 2}: If Alice choses a lower point than $\frac{1}{4}$. Bob can know employ the same strategy as before,
    he can move a little bit more to the edge than Alice. This makes Charlize move into the middle. Since the middle is now
    bigger than if Alice had chosen $\frac{1}{4}$, Charlize gets more of the mid-part and Alice looses some of here share.
    $\Rightarrow$ Therefore, Alice should not move right of $\frac{1}{4}$. the only reasonable option therefore is, to stay at
    $\frac{1}{4}$.
\end{document}
