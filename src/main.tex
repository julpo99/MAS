%! Author = julianramondo
%! Date = 09/11/2023

% Preamble
\documentclass[11pt]{article}
\title{Homework Assignment 1}

% Packages
\usepackage{amsmath}
\usepackage{multirow}
\usepackage{pgfplots}
\pgfplotsset{compat=1.12}


% Document
\begin{document}
    \maketitle


    \section{Game Theory: Optimality Concepts and Nash Equilibrium}

    \subsection{Odd or even game}

    \subsubsection{Answer 1:}
    Payoff matrix:
% Please add the following required packages to your document preamble:
    \begin{table}[h]
    \centering
    \begin{tabular}{llllll}
                       &                        & \multicolumn{2}{c}{B}                                 &  &  \\
                       &                        & \multicolumn{1}{c}{1}     & 2                         &  &  \\ \cline{3-4}
    \multirow{2}{*}{A} & \multicolumn{1}{r|}{1} & \multicolumn{1}{l|}{-2, 2} & \multicolumn{1}{l|}{3, -3} &  &  \\ \cline{3-4}
                       & \multicolumn{1}{l|}{2} & \multicolumn{1}{l|}{3, -3} & \multicolumn{1}{l|}{-4, 4} &  &  \\ \cline{3-4}
                       &                        &                           &                           &  &
    \end{tabular}
    \end{table}
    \subsection{Answer 2:}
    Regret minimisation strategies in terms of pure strategies:
    \begin{table}[h]
    \centering
    \begin{tabular}{llllll}
                       &                        & \multicolumn{2}{c}{B}                                                                                                                             &  &  \\
                       &                        & \multicolumn{1}{c}{1}                                                   & 2                                                                       &  &  \\ \cline{3-4}
    \multirow{2}{*}{A} & \multicolumn{1}{r|}{1} & \multicolumn{1}{l|}{\begin{tabular}[c]{@{}l@{}}5 0\end{tabular}} & \multicolumn{1}{l|}{\begin{tabular}[c]{@{}l@{}}0 5\end{tabular}} &  &  \\ \cline{3-4}
                       & \multicolumn{1}{l|}{2} & \multicolumn{1}{l|}{\begin{tabular}[c]{@{}l@{}}0 7\end{tabular}} & \multicolumn{1}{l|}{\begin{tabular}[c]{@{}l@{}}7 0\end{tabular}} &  &  \\ \cline{3-4}
                       &                        &                                                                         &                                                                         &  &
    \end{tabular}
    \end{table}
    \subsection{Answer 3:}
    Regret minimisation strategies in terms of mixed strategies:
    \begin{figure}[h]
    \begin{tikzpicture}[>=latex]
        \centering
        \begin{axis}
            [
            axis x line=center,
            axis y line=center,
            width={\linewidth},
            ytick={1,2,...,7},
            xtick={0.1,0.2,...,1},
            xlabel={$p$},
            ylabel={$R_A$},
            xlabel style={below right},
            ylabel style={above left},
            xmin=0,
            xmax=1.1,
            ymin=0,
            ymax=8,
            domain=0:1]
            \addplot+[mark=] {5*x};
            \addplot+[mark=] {7-7*x)};
            \node[red,scale=6] at (axis cs:7/12,35/12) {.};
            \draw[red, dashed] (axis cs:7/12,35/12) -- (axis cs:7/12,0);
            \node[above right, red] at (axis cs:7/12,0) {$\cfrac{7}{12}$};
        \end{axis}
    \end{tikzpicture}
    \end{figure}

    \subsection{Answer 4:}

\section{1.3}
\subsection*{1.3.1}
Utility is given by:
\begin{equation}
    \begin{split}
        u_i(q_1,q_2)&=(\alpha -\beta(q_1+q_2) - c_i)*q_i\\
                    &=\alpha*q_i -\beta*q_1^2 -\beta*q_2*q_1 - c_i*q_2
    \end{split}        
\end{equation}
where either $q_1$ or $q_2$ is fixed and $\alpha, \beta>0$.
Lets say company $1$ has to respond to company $2$, then the maximum profit is given by the maximum utility.
\begin{align}
    & \cfrac{\partial u_1}{\partial q_1}=\alpha - 2\beta*q_1 - \beta*q_2 - c_1\\
    & \cfrac{\cfrac{\partial u_1}{\partial q_1}}{\partial q_1}=-2\beta
\end{align}
Where we get the maximum by calculating the x-axis intersection of the derivative (while making sure it is a maximum by
looking at the second derivative)
\begin{align}
    &\cfrac{\partial u_1}{\partial q_1}=0 \leftrightarrow \alpha - 2\beta*q_1 - \beta*q_2 - c_1=0\\
    & q_1 = \cfrac{-\alpha+\beta*q_2 + c_1}{-2\beta}
    \label{eqn:max_q_1}
\end{align}
The value seen in equation~\ref{eqn:max_q_1} is therefore the best response of company one if company two already
given their quantity. Similarly the best result for company two ($q_2$) can be derived:
\begin{equation*}
    q_2 = \cfrac{-\alpha+\beta*q_1 + c_1}{-2\beta}
\end{equation*}
\subsection*{1.3.2}
In an infinite game, the two best responses will change for each company until they join in an Equilibrium. The actual values
will be dependent on the constant parameters $\alpha, \beta, c_1,$ and $c_2$.
\begin{figure}[h]
    \begin{tikzpicture}[>=latex]
        \centering
        \begin{axis}
            [
            axis x line=center,
            axis y line=center,
            width={\linewidth},
            ytick={},
            xtick={},
            xlabel={$q_2$},
            ylabel={$q_1$},
            xlabel style={below right},
            ylabel style={above left},
            xmin=0,
            xmax=6,
            ymin=0,
            ymax=6,
            domain=0:6]
            \addplot+[mark=] {-2*x+5};
            \addplot+[mark=] {-0.5*x+2.5};
            \node[red,scale=4] at (axis cs:5/3,5/3) {.};
            \node[above right, red] at (axis cs:5/3,5/3) {Nash Equilibrium};
        \end{axis}
    \end{tikzpicture}
    \caption{Illustration of the quantities each company will produce. Notice that the lines are similar only on different axis,
    which is only the case if they have the same production cost $c_i$.}
    \end{figure}
\section*{1.4}
\subsection*{1.4.1}
We can give the utility as the area that Charlize will serve (equation~\ref{eqn:utility_c})
\begin{align}
    \begin{split}
        u_c(a, b)&=(c+((b-c)/2)) - (c+(a-c)/2)\\
        & = 0.5b - 0.5c - 0.5a + 0.5c\\
        & = (b-a)/2
    \end{split}
    \label{eqn:utility_c}    
\end{align}
Therefore, the area Charlize will cover is maximized as long as she choses her position between Alice and Bob.
\subsection*{1.4.2}
We can make a case distinction. The area we get between Alice and Bob will be exactly half of the area between them.
Once the area that is available to us if we would go in-between Alice and Bob becomes smaller than the area to the right of
Bob, we just go to the right of Bob. Otherwise we stay in the middle.
\begin{equation}
    BR_c(a,b)=\left\{
    \begin{array}{lll}
        &\text{directly right of b } &\mbox{if } \frac{(b-a)}{2}\leq(1-b)\\
        &\mbox{between a and b}      &\text{otherwise}
    \end{array}
    \right.
\end{equation}
\subsection*{1.4.3}
We want to maximize our expected utility (served area) considering that one more seller (Charlize) will also come and maximize
her expected utility. For this, we need a position in which Charlize will move between us an Alice and the area between us and
Alice is minimized. This is exactly the case (see above), when the following condition is true:
\begin{equation}
    1-b = (b-a)/2
\end{equation}
Therefore, Bob will position himself exactly at $0,7$ to maximize his utility while considering that Charlize will still show up.

\subsection*{1.4.4}
She should set up her place exactly at $\cfrac{1}{4}$ or $\cfrac{3}{4}$ (because of symmetry).
\end{document}
